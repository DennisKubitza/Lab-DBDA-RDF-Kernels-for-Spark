%%%%%%%%%%%%%%%%%%%%%%%%%%%%%%%%%%%%%%%%%%%%%%%%%%%%%%%%%%%%%%%%%%%
%                 Erstellt von Matthias Duch 2017       
%				  Mehrfach abgeändert von Dennis Kubitza 2017
%%%%%%%%%%%%%%%%%%%%%%%%%%%%%%%%%%%%%%%%%%%%%%%%%%%%%%%%%%%%%%%%%%%
\usepackage[ngerman]{babel}\usepackage[babel=true]{microtype}
\usepackage[utf8]{inputenc}%Without utf8x
\usepackage[T1]{fontenc}\usepackage{lmodern}\usepackage{url}\usepackage{color}\usepackage{tikz}\usetikzlibrary{shadows,arrows,shapes,calc}
\usepackage{amsmath,amssymb,amsfonts,amsthm,amsbsy}%$\varmathbb{R}$
\usepackage{graphicx} %\includegraphics[width=0.7\textwidth]{bild.jpg}
\usepackage{placeins} %for \FloatBarrier
\usepackage{setspace}
\usepackage{comment}
\usepackage{hyperref}
\usepackage{listings}
\usepackage{caption}
\usepackage{multicol}
\usepackage{xcolor}
\usepackage{smartdiagram}
\usepackage{dcolumn}%für Stargazerimport

%%%%%%%%%%%%%%%%%%%%%%%%%%%%%%%%%%%%%%%%%%%%%%%%%%%%%%%%%%%%%%%%%%%%%%%%%%%
%                         Angaben auf den Folien                          %
%%%%%%%%%%%%%%%%%%%%%%%%%%%%%%%%%%%%%%%%%%%%%%%%%%%%%%%%%%%%%%%%%%%%%%%%%%%
\title{"Graph Kernels for RDF Data"$^\text{*}$- Summary and implementation ideas}
\author{Dennis Oliver Kubitza}
\date{\today}



%%%%%%%%%%%%%%%%%%%%%%%%%%%%%%%%%%%%%%%%%%%%%%%%%%%%%%%%%%%%%%%%%%%
%                            Themeneinstellungen                  %
%%%%%%%%%%%%%%%%%%%%%%%%%%%%%%%%%%%%%%%%%%%%%%%%%%%%%%%%%%%%%%%%%%%
%remove the icon
\setbeamertemplate{bibliography item}{}
%remove line breaks
\setbeamertemplate{bibliography entry title}{}
\setbeamertemplate{bibliography entry location}{}
\setbeamertemplate{bibliography entry note}{}

\newcommand*\keystroke[1]{\tikz[baseline=(key.base)]\node[draw, fill=white, drop shadow={shadow xshift=0.25ex,shadow yshift=-0.25ex,fill=black,opacity=0.75}, rectangle, rounded corners=2pt, inner sep=1pt, line width=0.5pt, font=\scriptsize\sffamily ](key) {#1\strut};}
\newsavebox{\codebox}% For saving code

\usetheme{Madrid}
\usecolortheme{beaver}
\makeatletter
\setbeamertemplate{footline}{
  \hbox{%
  \begin{beamercolorbox}[wd=1\paperwidth,ht=2.25ex,dp=1ex,center]{author in head/foot}%
    \usebeamerfont{title in head/foot}\inserttitle
  \end{beamercolorbox}}%
}
\setbeamertemplate{headline}
{
  \leavevmode%
  \hbox{%
  \begin{beamercolorbox}[wd=.333333\paperwidth,ht=2.25ex,dp=1ex,center]{author in head/foot}%
    \usebeamerfont{author in head/foot}\insertsection
  \end{beamercolorbox}%
  \begin{beamercolorbox}[wd=.333333\paperwidth,ht=2.25ex,dp=1ex,center]{title in head/foot}%
    \usebeamerfont{title in head/foot}\insertsubsection
  \end{beamercolorbox}%
  \begin{beamercolorbox}[wd=.333333\paperwidth,ht=2.25ex,dp=1ex,right]{date in head/foot}%
    \usebeamerfont{date in head/foot}\insertshortdate{}\hspace*{2em}
    \insertframenumber{} / \inserttotalframenumber\hspace*{2ex} 
  \end{beamercolorbox}}%
  \vskip0pt%
}

\newcommand\blfootnote[1]{%
  \begingroup
  \renewcommand\thefootnote{}\footnote{#1}%
  \addtocounter{footnote}{-1}%
  \endgroup
}
\addto\captionsngerman{%Abkürzen der Überschriften
\renewcommand{\figurename}{Abb.}
\renewcommand{\tablename}{Tab.}
}
\usepackage{framed}
\newcommand\result[1]{\begin{framed}#1\end{framed}}

\definecolor{themenrot}{RGB}{163,0,0}
\definecolor{themengruen}{RGB}{0,82,41}
\input{settingsListing}
\captionsetup{justification=centering}
\setbeamercolor*{item}{fg=themengruen}% make bullets red fit to greentheme
\newcommand{\code}[1]{\texttt{#1}}
\newcommand{\bcode}[1]{\texttt{\textbf{#1}}} %fetter Code
\newcommand{\ccode}[1]{\begin{center}\texttt{\textbf{\Large{#1}}}\end{center}} 
\makeatother
\usecolortheme{spruce}% gruenes Theme




%%%%%%%%%%%%%%%%%%%%%%%%%%%%%%%%%%%%%%%%%%%%%%%%%%%%%%%%%%%%%%%%%%%%%%%%%%%
%  Einstellungen fuer die Uebersicht am Anfang jedes Kapitels             %
%%%%%%%%%%%%%%%%%%%%%%%%%%%%%%%%%%%%%%%%%%%%%%%%%%%%%%%%%%%%%%%%%%%%%%%%%%%
\AtBeginSection[]{
\begin{frame}<beamer|handout>
\frametitle{Content Overview}
\tableofcontents[currentsection]%,hideallsubsections]
\end{frame}}
%Übersicht am Anfang einer Subsection
\AtBeginSubsection[]{
\begin{frame}<beamer|handout>
\frametitle{Content Overview}
\tableofcontents[currentsection,hideothersubsections,sectionstyle=show/hide,        subsectionstyle=show/shaded/hide]
\end{frame}}



%%%%%%%%%%%%%%%%%%%%%%%%%%%%%%%%%%%%%%%%%%%%%%%%%%%%%%%%%%%%%%%%%%%%%%%%%%%
%                         Design fuer Aufzaehlungen                       %
%%%%%%%%%%%%%%%%%%%%%%%%%%%%%%%%%%%%%%%%%%%%%%%%%%%%%%%%%%%%%%%%%%%%%%%%%%%
\setbeamertemplate{sections/subsections in toc}[square]
\setbeamertemplate{enumerate items}[square]
\setbeamertemplate{itemize items}[square]
