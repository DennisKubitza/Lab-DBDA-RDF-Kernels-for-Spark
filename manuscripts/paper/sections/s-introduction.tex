
%------------------------------------------------------------------------------
\section{Introduction}
%------------------------------------------------------------------------------
While each every Machine Learning Task is defined by its Input Set, it's Set of valid models and the expected behaviour of the learning Agent, some algorithms exist that solve problems under such general assumtpions that almost any Data-dependet Problem can be reduced to fit their requirements. Kernel-based Machine learning methods don't require any specific structure for the Data, solomly that a scalar valued function exists, suitable for summarizing a Observation or Subobservation as a single value. We call such a function a kernel Functions. Before stating a mathematic exact definition, applyable to even the most general settings we are taking a look on some examples where Kernel-Based Machine learning is applied to RDF. 

\paragraph{Kernel Examples}
\begin{itemize}
\item XXX
\item XXX
\item XXX
\end{itemize}

