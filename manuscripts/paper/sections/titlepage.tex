
\maketitle
\begin{center}\large
\href{https://github.com/DennisKubitza/Lab-DBDA-RDF-Kernels-for-Spark}{Project References on Github} \\
Lab Report \footnote{as Part of the Examination of Modul 4223, Master of Computer Science, University of Bonn} 
\end{center}

\vspace{0.5cm}
\renewcommand{\baselinestretch}{1.3}\normalsize

\pagenumbering{arabic}
\setcounter{page}{1}
\thispagestyle{empty}

\begin{center}\textbf{Abstract}\end{center}
\begin{abstract}
\noindent Many implemented machine learning Algorithms strongly depend on the specific structure of the observed and unobserved Data, which forces users to fit their Observations in a perticular predefined setting or reimplement the algorithms to fit their requirements. For dynamic Data models like \href{https://w3.org/RDF}{Resouce Description Frameworks}, that operate on schema-free Structures, one class of Algorithms is naturally well suited to compromise both approaches: Kernel Based Algorithms. We follow the examinations of \citet{mainsource} and implement their proposed Graph-Kernels for the usage in \href{https://spark.apache.org/research.html}{Apache Spark}, especially for further usage in the \href{http://sansa-stack.net/}{Semantic Analytics Stack (SANSA)}. Our implementation combines different approaches from Graph Combinatorics, Data-Mining and Big Data Analysis to ensure scalability in storage and computational performance.  \newline
\end{abstract}


\newpage
\tableofcontents
\newpage
